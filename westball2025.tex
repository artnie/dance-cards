\documentclass[
	12pt,
	%draft, % show overfull hboxes
	]{scrartcl}

\usepackage[    % set up dimensions
	a6paper, 
	landscape,
	left=1.45cm,   
	right=1.45cm,  
	top=1.6cm,
	bottom=1.6cm, 
	%showframe,
	nomarginpar,  % this removes the right margin, centering the text in the middle
	noheadfoot,  % this removes the head and foot margin
	]{geometry}

%\pagestyle{empty}

\newif\iffast
\fastfalse

\usepackage{graphicx}
\usepackage{tikz}
\usetikzlibrary{external}
\usepackage{eso-pic}

% Fonts and encodings
\usepackage[nf]{coelacanth}
\usepackage[T1]{fontenc}

\usepackage{longtable} % Um Tabellen zu machen, die mehrere Seiten lang sind und einen passenden Seitenumbruch haben.
\usepackage{multicol}
%\usepackage[singlespacing]{setspace} % Zeilenabstand
\usepackage{nicefrac} % für besser aussehende Brüche

\usepackage[object=vectorian]{pgfornament}

% pstricks vs pgf/tikz:
% Originally we used pstricks to draw the background ornaments.
% This had the disadvantage of having to use `latex -> dvips -> ps2pdf` with the correct options to get the background
% graphics to display in the right orientation. The necessary options also varied between MiKTeX and TeXLive :-(.
%
% Since i have more experience with tikz, i decided to use that, because it was easier to position the
% elements (they weren't placed quite correctly with the pstricks version.).

% pgfornament vs eps files
% I wanted to use pgfornament at first because it has everything necessary in one easy package.
% Unfortunately the graphics in pgfornament end up having quite thick strokes. This does not look good at the scales
% we're using the patterns.
% So i downloaded the vectorian files and extracted the patterns manually with Inkscape. This probably leads to a slowdown though.


\AddToShipoutPictureBG{%
\begin{tikzpicture}[remember picture, overlay]
	% pgfornament does give very simple and nice vectorian graphics for tikz.
	% The problem is that the lines are too thick and figure 41 is also slightly cropped.
	% pst-vectorian works very well.
	% we're now using the original files from vectorian exported as eps file
	
	% DEBUGGING grid
	%\draw[line width=0.75pt, color=teal, step=.5cm, xshift=0.08cm, yshift=0.43cm] (current page.south west) grid ++(14.85cm,10.5cm);
	% DEBUGGING corners
	%\node [circle, fill=red, minimum size=1pt, xshift=0.0cm, yshift=-0.0cm, scale=0.2] at (current page.north west) {};
	%\node [circle, fill=red, minimum size=1pt, xshift=0.5cm, yshift=-0.5cm, scale=0.2] at (current page.north west) {};
	
	% TODO: This slows down compilation a lot.
	%       Maybe find a faster way to do this.
	\iffast
	\else
	\node[anchor=north west, xshift=0.5cm, yshift=-0.5cm, inner sep=0] at (current page.north west){\includegraphics[scale=0.8, angle=0, origin=c]{img/corner_black}};
	\node[anchor=north east, xshift=-0.5cm, yshift=-0.5cm, inner sep=0] at (current page.north east){\includegraphics[scale=0.8, angle=0, origin=c]{img/corner2_black}};
	\node[anchor=south west, xshift=0.5cm, yshift=0.5cm, inner sep=0] at (current page.south west){\includegraphics[scale=0.8, angle=180, origin=c]{img/corner2_black}};
	\node[anchor=south east, xshift=-0.5cm, yshift=0.5cm, inner sep=0] at (current page.south east){\includegraphics[scale=0.8, angle=180, origin=c]{img/corner_black}};
	\fi
	
	
	% Original pgfornament style:
%	\node[anchor=north west, xshift=0.5cm, yshift=-0.5cm,inner sep=0] at (current page.north west){\pgfornament[scale=0.5]{41}};
%	\node[anchor=north east, xshift=-0.5cm, yshift=-0.5cm, inner sep=0] at (current page.north east){\pgfornament[scale=0.5,symmetry=v]{41}};
%	\node[anchor=south west, xshift=0.5cm, yshift=0.5cm, inner sep=0] at (current page.south west){\pgfornament[scale=0.5,symmetry=h]{41}};
%	\node[anchor=south east, xshift=-0.5cm, yshift=0.5cm, inner sep=0] at (current page.south east){\pgfornament[scale=0.5,symmetry=c]{41}};
	
	% Background as an image, for example a parchment like texture
	%\tikz[remember picture, overlay] \node[opacity=0.5, inner sep=0pt] at (current page.center){\includegraphics[width=\paperwidth,height=\paperheight]{example-image}};
	
\end{tikzpicture}
}

\newcommand{\danceinfo}[1]{%
	{\raggedleft\footnotesize{#1}\par}
}
\newcommand{\dancename}[1]{
	% Dimension are manually chosen so they don't collide with the ornaments in the corners.	
	\begin{tikzpicture}[remember picture, overlay]
%		\node[] at (0, 0) (title) {Text};		
		\node[anchor=north, yshift=-0.5cm, inner sep=0, text width=10.5cm, align=center] at (current page.north){\Huge{#1}};
	\end{tikzpicture}
}
\newcommand{\origininfo}[3]{
	% Dimension are manually chosen so they don't collide with the ornaments in the corners.
	\begin{tikzpicture}[remember picture, overlay]
		% setting the font size in the node options, since it doesn't seem to 
		% take effect when used in the node text. 
		\node[anchor=south west, xshift=1.3cm, yshift=1.15cm, inner sep=0, align=left, font={\scriptsize}] at (current page.south west){Choreographie:\\{#1}};
		\node[anchor=south east, xshift=-1.2cm, yshift=1.15cm, inner sep=0, align=right, font={\scriptsize}] at (current page.south east){Musik:\\ {#2}\\ {#3}};
	\end{tikzpicture}
}
\newcommand{\danceeasymarker}{
	\iffast
	\else
		\begin{tikzpicture}[remember picture, overlay]
			\node[anchor=south, xshift=0cm, yshift=0.5cm, inner sep=0] at (current page.south){\pgfornament[width=4cm]{80}};
		\end{tikzpicture}
	\fi
}
\newcommand{\dancemediummarker}{
	\iffast
	\else
		\begin{tikzpicture}[remember picture, overlay]
			\node[anchor=south, xshift=0cm, yshift=0.5cm, inner sep=0] at (current page.south){\pgfornament[width=5cm]{83}};
		\end{tikzpicture}
	\fi
}
\newcommand{\dancedifficultmarker}{
	\iffast
	\else
		\begin{tikzpicture}[remember picture, overlay]
			\node[anchor=south, xshift=0cm, yshift=0.4cm, inner sep=0] at (current page.south){\pgfornament[width=5cm]{84}};
		\end{tikzpicture}
	\fi
}
\newcommand{\danceinstructionsbegin}{\begin{longtable}{p{1cm}p{9.8cm}}}
\newcommand{\danceinstructionsend}{\end{longtable}}
\newcommand{\danceinstructionsel}{~ & ~ \\}

\setlength{\topskip}{12pt}
\setlength{\parskip}{0pt}

\pagestyle{empty}

\begin{document}

%%%%%%%%%%%%%%%%%%%%%%%%%%%%%%%%%%%%%%%%%%%%%%%%%%%%%%%%%%%%%%%%%%%%%%%%%%%%%%%%
% A' Rovin
%%%%%%%%%%%%%%%%%%%%%%%%%%%%%%%%%%%%%%%%%%%%%%%%%%%%%%%%%%%%%%%%%%%%%%%%%%%%%%%%

\iffalse\subsection{A' Rovin}\fi % fool texstudio into displaying subsections
\dancename{A' Rovin}
\origininfo{Cecil Sharp }{A-Roving / Faithless Nancy Dawson}{Traditional/Anna Bidder}
\danceinfo{Longway for as many as will}
\danceinstructionsbegin
1 -- 8 	& Paar 1 Lead Down\& Cast Up (lebhaft gesprungen)\\
9 -- 16	& Paar 2 Lead Up \& Cast Down (lebhaft gesprungen)\\
\danceinstructionsel
1 -- 8 	& Dos-á-dos\\
\danceinstructionsel
1 -- 2	& Rechtsschultrig nach außen hüpfen \& \nicefrac{1}{2} Drehung \\
3 -- 4  & Linksschultrig nach außen hüpfen \\
5 -- 8  & Eindrehen rechtsschultrig auf einander zu\\
\danceinstructionsel
1 -- 8 	& Rechte Hand reichen zur \nicefrac{3}{4} Kette
\danceinstructionsend
\danceeasymarker

\newpage

%%%%%%%%%%%%%%%%%%%%%%%%%%%%%%%%%%%%%%%%%%%%%%%%%%%%%%%%%%%%%%%%%%%%%%%%%%%%%%%%
% Blaue Flagge
%%%%%%%%%%%%%%%%%%%%%%%%%%%%%%%%%%%%%%%%%%%%%%%%%%%%%%%%%%%%%%%%%%%%%%%%%%%%%%%%

\dancename{Blaue Flagge}
\iffalse\subsection{Blaue Flagge}\fi % fool texstudio into displaying subsections
\danceinfo{Kreistanz, Herr innen, Dame außen, ¾ Takt, "Hoch" auf 1 \\Partner anschauen, rechte Handfassung\\}
%\origininfo{???}{???}{}
\danceinstructionsbegin
1--6 	& Balance vor zurück\\
7--12 	& Platzwechsel mit Drehung\\
1--12 	& Wiederholung\\
\danceinstructionsel
1--6 	& Balance vor,  zurück\\
7--9 	& Herr Schritt vor, Dame eindrehen zur Kiekbuschfassung\\
10--12	& Gemeinsam einen Schritt zurück\\
1--3 	& Gemeinsam einen Schritt nach vorne\\
4--6 	& Aus der Kiekbuschfassung lösen (Herr Handwechsel)\\
\danceinstructionsel
7--9 	& Beide kicken mit innerem Fuß gegen die Kreisrichtung\\
10--12 	& Dame dreht unter dem li. Arm des Herrn zum nächsten 
\danceeasymarker
\danceinstructionsend

\newpage

%%%%%%%%%%%%%%%%%%%%%%%%%%%%%%%%%%%%%%%%%%%%%%%%%%%%%%%%%%%%%%%%%%%%%%%%%%%%%%%%
% Branle du Rats
%%%%%%%%%%%%%%%%%%%%%%%%%%%%%%%%%%%%%%%%%%%%%%%%%%%%%%%%%%%%%%%%%%%%%%%%%%%%%%%%

\dancename{Branle du Rats}
\iffalse\subsection{Branle du Rats}\fi % fool texstudio into displaying subsections
%\origininfo{???}{???}{}
\danceinfo{Longway for as many as will (Kein Fortschritt)}
\danceinstructionsbegin
1--8 	& Alle Double links, kick, Double rechts, kick\\
9--16 	& Wiederholen\\
1--2 	& Die Tänzer springen mit dem linken Fuß schräg links aufeinander zu\\
3--4 	& Dann springen sie mit dem rechten Fuß schräg aufeinander zu, stehen Linksschultrig nebeneinander\\
5--6 	& Mit einem weiteren Sprung mit links gehen sie Rücken an Rücken aneinander vorbei\\
7--8 	& Mit dem letzten rechten Schritt drehen sie sich fertig auf die gegenüberliegende Grundlinie\\
\danceeasymarker
\danceinstructionsend
\textit{Wiederholen und dabei die Geschwindigkeit der Musik anpassen}

\newpage

%%%%%%%%%%%%%%%%%%%%%%%%%%%%%%%%%%%%%%%%%%%%%%%%%%%%%%%%%%%%%%%%%%%%%%%%%%%%%%%%
% Candles in the Dark
%%%%%%%%%%%%%%%%%%%%%%%%%%%%%%%%%%%%%%%%%%%%%%%%%%%%%%%%%%%%%%%%%%%%%%%%%%%%%%%%

\dancename{Candles in the Dark}
\origininfo{Loretta Holz (2006)}{Candles in the Dark}{von Jonathan Jensen}
\iffalse\subsection{Candles in the Dark}\fi % fool texstudio into displaying subsections
\danceinfo{Longway for as many as will\\¾ Takt, Walzerschritte}
\danceinstructionsbegin
1--12 	& H1 D1 geführte Half-Figure Eight durch das andere Paar\\
1--12 	& H1 D2 geführte Half-Figure Eight durch das andere Paar\\
1--12 	& H2 D1 geführte Half-Figure Eight durch das andere Paar\\
1--12 	& H2 D2 geführte Half-Figure Eight durch das andere Paar\\
\danceinstructionsel
1--12 	& Mirror Dos-à-Dos auf der Linie - Paar 2 innen \\
1--12 	& ganzer Setkreis\\
\danceinstructionsel
1--12 	& Mirror Dos-à-Dos auf der Linie – Paar 1 innen\\
1--12 	& 1 ½ Ronden mit dem Partner\\
\dancemediummarker
\danceinstructionsend

\newpage

%%%%%%%%%%%%%%%%%%%%%%%%%%%%%%%%%%%%%%%%%%%%%%%%%%%%%%%%%%%%%%%%%%%%%%%%%%%%%%%%
% Chapelloise
%%%%%%%%%%%%%%%%%%%%%%%%%%%%%%%%%%%%%%%%%%%%%%%%%%%%%%%%%%%%%%%%%%%%%%%%%%%%%%%%

\dancename{Chapelloise}
\iffalse\subsection{Chapelloise}\fi % fool texstudio into displaying subsections
\origininfo{Traditionell}{Jochen Eßrich}{Saitenhieb + Lebenswecker}
\danceinfo{Kreistanz, Herren innen, gegen den UZS}
\danceinstructionsbegin
\textbf{I}&\\
1--4 	& Vier Schritte vor, auf 4 Drehung und Handwechsel\\
5--8 	& Vier Schritte rückwärts\\
1--8 	& Eben so zurück\\
\textbf{II}&\\
1--2 	& Zueinander hüpfen, „flirten“\\
3--4 	& Außeinander hüpfen, nach außen respektive innen schauen\\
5--8 	& Platzwechsel, Dame geht vor dem Herren vorbei\\
1--2 	& Zueinander\\
3--4 	& Außeinander\\
5--8 	& Die Dame dreht unter dem linken Arm des Herren nach \\
		& hinten (über ihre rechte Schulter)
\danceeasymarker
\danceinstructionsend

\newpage

%%%%%%%%%%%%%%%%%%%%%%%%%%%%%%%%%%%%%%%%%%%%%%%%%%%%%%%%%%%%%%%%%%%%%%%%%%%%%%%%
% Circassian Circle
%%%%%%%%%%%%%%%%%%%%%%%%%%%%%%%%%%%%%%%%%%%%%%%%%%%%%%%%%%%%%%%%%%%%%%%%%%%%%%%%

\dancename{Circassian Circle}
\iffalse\subsection{Circassian Circle}\fi % fool texstudio into displaying subsections
\origininfo{Maud Karpeles (1918)}{Tanzfieber}{Teufelskreis}
\danceinfo{Kreistanz, Herr/Dame, durchgefasst}
\danceinstructionsbegin
1 -- 16 & Alle 2x Double vor und zurück\\
\danceinstructionsel
1 -- 8 	& Damen Double vor \& zurück, (H klatschen)\\
1 -- 8  & Herren eben so (D klatschen), zur linken Dame wenden \\
\danceinstructionsel
1 -- 8  & Drehen: Rechte Füße aneinander. Rechte Hand auf linke Schulter d. Partners. Linke Hände darunter fassen. Zurücklehnen. Mit dem linken Fuß abstoßen, beliebig oft drehen.\\
9 -- 16 & Eben so ggn. d. UZS, Ende in Kiekbuschfassung \\
1 -- 12 & In Kiekbuschfassung auf der Kreisbahn entlangschreiten\\
13 -- 16& Dame dreht unter dem rechten Arm d. Herren zur rechten.\\
& Fortschritt der Dame. Im Kreis durchfassen.
\danceeasymarker
\danceinstructionsend


\newpage

%%%%%%%%%%%%%%%%%%%%%%%%%%%%%%%%%%%%%%%%%%%%%%%%%%%%%%%%%%%%%%%%%%%%%%%%%%%%%%%%
% Emperor of the Moon
%%%%%%%%%%%%%%%%%%%%%%%%%%%%%%%%%%%%%%%%%%%%%%%%%%%%%%%%%%%%%%%%%%%%%%%%%%%%%%%%

\dancename{Emperor of the Moon}
\iffalse\subsection{Emperor of the Moon}\fi % fool texstudio into displaying subsections
\origininfo{Playford}{Playford}{Emperor of the Moon}
\danceinfo{Longway for as many as will}%\\wird schneller}
\danceinstructionsbegin
1 -- 8 	& Set \& Turn links, beim Turn weit zurücklaufen\\
9 -- 12 	& Double vorwärts man trifft sich wieder in der Mitte\\
13 -- 16 	& Set rechts\\
\danceinstructionsel
& \textit{Folgendes ist flüssig:}\\
1 -- 8 	& Paar 1 cast down, geht auf Platz von Paar 2 \\
ab 3  	& P2 \textbf{erstes} Kreuzen: auf Platz des Konterpartners\\
& \textit{Damen kreuzen vor dem Herrn} \\
auf 8   & Doppelhandfassung der Konter H1+D2 und D1+H2 \\
1 -- 4 	& Konterpaare tanzen je ganze Ronde, mit 1-ern innen \\
5 -- 8  & P2 \textbf{zweites} Kreuzen: 1-er werfen 2-er nach oben \\
& mit dem Schwung aus der Ronde (Fortschritt)\\
\dancemediummarker
\danceinstructionsend

\newpage

%%%%%%%%%%%%%%%%%%%%%%%%%%%%%%%%%%%%%%%%%%%%%%%%%%%%%%%%%%%%%%%%%%%%%%%%%%%%%%%%
% Heptathlon Jig
%%%%%%%%%%%%%%%%%%%%%%%%%%%%%%%%%%%%%%%%%%%%%%%%%%%%%%%%%%%%%%%%%%%%%%%%%%%%%%%%

\dancename{Heptathlon Jig}
\iffalse \subsection{Heptathlon Jig}\fi % fool texstudio into displaying subsections
\origininfo{Alan W. White}{I lost my Love / The Old Favorite}{}
\danceinfo{7 Tänzer, in einem 'H' aufgestellt}
\danceinstructionsbegin
1 -- 8 	& Stern zu dritt oben rechts\\
1 -- 8 	& Stern zu dritt oben links\\
1 -- 8 	& Stern zu dritt unten rechts\\
1 -- 8 	& Stern zu dritt unten links\\
\danceinstructionsel
1 -- 16 & Heckenacht der mittleren Reihe\\
1 -- 8 	& Die Mitte tanzt mit rechts oben 1½ Ronden\\
~		& \textit{bzw. etwas beschwingtes, das in einem Platzwechsel resultiert}\\
1 -- 8 	& Set \& Turn links auf die nächste Außenposition\\
\danceeasymarker
\danceinstructionsend

\newpage

%%%%%%%%%%%%%%%%%%%%%%%%%%%%%%%%%%%%%%%%%%%%%%%%%%%%%%%%%%%%%%%%%%%%%%%%%%%%%%%%
% Hole in the Wall
%%%%%%%%%%%%%%%%%%%%%%%%%%%%%%%%%%%%%%%%%%%%%%%%%%%%%%%%%%%%%%%%%%%%%%%%%%%%%%%%

\dancename{Hole in the Wall}
\iffalse\subsection{Hole in the Wall}\fi % fool texstudio into displaying subsections
\origininfo{Playford}{Henry Purcell}{Hornpipe from Abdelazar}
\danceinfo{Longway for as many as will\\¾ Takt, Walzerschritte}
\danceinstructionsbegin
1--12 	& Paar 1 Cast down \& Lead up\\
1--12 	& Paar 2 Cast up \& Lead down\\
~ & ~ \\
1--6 	& FC Platzwechsel rechte Hand, kurz in der Mitte verharren\\
7--12	& SC Platzwechsel linke Hand, kurz in der Mitte verharren\\
~ & ~ \\
1--6 	& Halber Setkreis\\
7--12	& Paar 1 Cast down und zieht Paar 2 hoch \\
\danceeasymarker
\danceinstructionsend

\newpage

%%%%%%%%%%%%%%%%%%%%%%%%%%%%%%%%%%%%%%%%%%%%%%%%%%%%%%%%%%%%%%%%%%%%%%%%%%%%%%%%
%	Indian Queen
%%%%%%%%%%%%%%%%%%%%%%%%%%%%%%%%%%%%%%%%%%%%%%%%%%%%%%%%%%%%%%%%%%%%%%%%%%%%%%%%

\dancename{Indian Queen}
\iffalse\subsection{Indian Queen}\fi % fool texstudio into displaying subsections
\origininfo{Playford}{Indian Queen}{}
\danceinfo{Longway for as many as will}
\danceinstructionsbegin
1 -- 4  & First Corner Set zum Partner dann zum Kontra\\
5 -- 8  & First Corner Turn rechts\\
9 -- 16 & First Corner Ronde\\
1 -- 8  & Second Corner Set \& Turn zum Partner, dann Kontra\\
9 -- 16 & Second Corner Ronde\\
\danceinstructionsel
1 -- 8  & Mühle rechts herum (rechte Hände) auf 8 Klatschen\\
9 -- 16 & Mühle links herum zurück, auf 8 Klatschen\\
\danceinstructionsel
1 -- 8  & Dos-à-dos\\
1 -- 8  & $\nicefrac{3}{4}$ Kette\\
\danceeasymarker
\danceinstructionsend

\newpage


%%%%%%%%%%%%%%%%%%%%%%%%%%%%%%%%%%%%%%%%%%%%%%%%%%%%%%%%%%%%%%%%%%%%%%%%%%%%%%%%
% Pavane d'Honneur
%%%%%%%%%%%%%%%%%%%%%%%%%%%%%%%%%%%%%%%%%%%%%%%%%%%%%%%%%%%%%%%%%%%%%%%%%%%%%%%%

\dancename{Pavane d'Honneur}
\iffalse\subsection{Pavane d'Honneur}\fi % fool texstudio into displaying subsections
%\origininfo{arg1}{arg2}{arg3}
\danceinfo{Reihentanz, Vorspiel \& Abschluss nicht Wiederholen, Blick nach oben}
\danceinstructionsbegin
1 -- 4 & (\textit{Vorspiel:} \textbf{P1} nach \textbf{links} 2 Simple, \textbf{P2} nach \textbf{rechts} 2 Simple)\\
\danceinstructionsel
1 -- 8 	& Simple schräg \textbf{vorwärts} links, rechts, links, rechts\\
9 -- 16 & Simple schräg \textbf{zurück} links, rechts, links, rechts\\
\danceinstructionsel
1 -- 8 	& \textit{Ausrichten:} \textbf{P1} nach \textbf{rechts} 4 Simple, \textbf{P2} nach \textbf{links} 4 Simple\\
9 -- 16 & Wiederholung zurück\\
\danceinstructionsel
1 -- 16 & Der Herr kniet ab, Die Dame umrundet ihn in Achteln\\
1 -- 16 & Der Herr umrundet die Dame in Achteln, sie dreht sich mit\\
\danceinstructionsel
1 -- 4 & (\textit{Abschluss:} \textbf{P1} nach \textbf{rechts} 2 Simple, \textbf{P2} nach \textbf{links} 2 Simple)
\dancemediummarker
\danceinstructionsend

\newpage

%%%%%%%%%%%%%%%%%%%%%%%%%%%%%%%%%%%%%%%%%%%%%%%%%%%%%%%%%%%%%%%%%%%%%%%%%%%%%%%%
% Red House
%%%%%%%%%%%%%%%%%%%%%%%%%%%%%%%%%%%%%%%%%%%%%%%%%%%%%%%%%%%%%%%%%%%%%%%%%%%%%%%%

\dancename{Red House}
\origininfo{Playford (1695)}{The Bare Necessities}{Red House}
\iffalse\subsection{Red House}\fi % fool texstudio into displaying subsections
\danceinfo{Longway for as many as will}

\danceinstructionsbegin
1 -- 8 	& \textbf{P1 Meet} \& Fallback, \textbf{P2 Fallback} \& Meet\\
1 -- 8  & P1 Set nach oben, unten, P1 Cast down, P2 lead up\\
1 -- 8 	& \textbf{P1 Meet} \& Fallback, \textbf{P2 Fallback} \& Meet\\
1 -- 8 	& P1 Set nach unten, oben, P1 Cast up, P2 lead down\\
\danceinstructionsel
& \textbf{Chasing:} \\
1 -- 12  & \textbf{Herr 1 wendet aus, Dame 1 jagd ihn}, er geht um Paar 2 herum.\\
13 -- 16 & Wenn er hinter D2 steht, rückt P2 nach oben auf\\
& und Paar 1 übernimmt den Platz von Paar 2 \\
1 -- 12  & \textbf{Dame 2 wendet aus, Herr 2 jagd sie}, sie geht um P2 herum. \\
13 -- 16 & Wenn sie hinter H1 steht, rückt P1 nach oben auf \\
& und Paar 2 übernimmt den Platz von Paar 1.\\
\danceinstructionsel
& \textbf{Hecke:} \\
1 -- 16 	& Herr 2 geht mit Paar 1 in eine Hecken-Acht\\
& beginnend mit Dame 1 rechtsschultrig \\
1 -- \textbf{12} & Dame 2 setzt die Hecke mit Paar 1 \textbf{zügig} fort\\
& beginnend mit Herr 1 linksschultrig \\
13 -- 16 	& Paar 1 wendet aus, Paar 2 schließt auf
\dancedifficultmarker
\danceinstructionsend


\newpage
%%%%%%%%%%%%%%%%%%%%%%%%%%%%%%%%%%%%%%%%%%%%%%%%%%%%%%%%%%%%%%%%%%%%%%%%%%%%%%%%
% Schiarazula Marazula
%%%%%%%%%%%%%%%%%%%%%%%%%%%%%%%%%%%%%%%%%%%%%%%%%%%%%%%%%%%%%%%%%%%%%%%%%%%%%%%%

\dancename{Schiarazula Marazula}
\iffalse\subsection{Schiarazula Marazula}\fi % fool texstudio into displaying subsections
\origininfo{Giorgio Mainerio (1574)}{Giorgio Mainerio}{Schiarazula Marazula}
\danceinfo{Kreistanz, Herr/Dame durchgefasst}
\danceinstructionsbegin
1 -- 8	& Doppel nach links, Doppel nach rechts (vorne kreuzend)\\
9 -- 16	& Wiederholen\\
\danceinstructionsel
1 		& Schritt \textbf{in den Kreis} (Herren rechts, Damen links)\\
2		& \textbf{Fremdem} Partner mit beiden Händen zuschnipsen.\\
3 -- 4	& Schritt mit anderem Fuß \& dem \textbf{eigenen} Partner schnipsen\\
5 -- 6	& Schritt mehr in den Kreis zum \textbf{fremden} Partner schnipsen\\
7 -- 8	& Nach außen drehen \& dreimal Klatschen (auf 6, 6\textsuperscript{und}, und 7)\\
& \textit{und wieder raus treten...}\\
1 -- 6 	& \textbf{Hinaus} zum fremden, eigenen, fremden Partner schnipsen\\\
7 -- 8	& Nach innen drehen \& dreimal Klatschen (auf 6, 6\textsuperscript{und}, und 7)
\danceeasymarker
\danceinstructionsend


\newpage
%%%%%%%%%%%%%%%%%%%%%%%%%%%%%%%%%%%%%%%%%%%%%%%%%%%%%%%%%%%%%%%%%%%%%%%%%%%%%%%%
% 	Siege of St. Malo
%%%%%%%%%%%%%%%%%%%%%%%%%%%%%%%%%%%%%%%%%%%%%%%%%%%%%%%%%%%%%%%%%%%%%%%%%%%%%%%%

\dancename{Siege of St. Malo}
\iffalse\subsection{Siege of St. Malo}\fi % fool texstudio into displaying subsections
\origininfo{Christine Feyerabend}{Siege of St. Malo}{}
\danceinfo{Longway for as many as will}
\danceinstructionsbegin
\danceinstructionsel
1 -- 8  & First Corner Dos-à-dos\\
9 -- 16 & Second Corner Dos-à-dos\\
\danceinstructionsel
1 -- 8  & $\nicefrac{3}{4}$ Kette\\
\danceinstructionsel
1 -- 8  & Set \& Turn rechts
\danceeasymarker
\danceinstructionsend

\newpage


%%%%%%%%%%%%%%%%%%%%%%%%%%%%%%%%%%%%%%%%%%%%%%%%%%%%%%%%%%%%%%%%%%%%%%%%%%%%%%%%
% Tourdion
%%%%%%%%%%%%%%%%%%%%%%%%%%%%%%%%%%%%%%%%%%%%%%%%%%%%%%%%%%%%%%%%%%%%%%%%%%%%%%%%

\dancename{Tourdion}
\origininfo{Arbeau (1589)}{Tourdion}{Pierre Attaignant (1530)}
\iffalse\subsection{Tourdion}\fi % fool texstudio into displaying subsections
\danceinfo{Kreistanz, Herr/Dame, durchgefasst}
\danceinstructionsbegin
1--4 	& Nach links, rechts vor und zurück wiegen\\
5--16 	& 3x Wiederholen \\
\danceinstructionsel
1--2 	& Die Herren geben/heben/werfen die rechts stehende Dame einen Platz nach links, danach \\
3--4 	& Vor und zurück wiegen\\
5--16 	& 3x Wiederholen \\
\danceinstructionsel
1--16 	& Grundschritt: Links, rechts, vor, zurück (x4)\\
%\danceinstructionsel
1--4 	& Die Damen geben/heben/werfen den links stehenden Herren einen Platz nach rechts, danach Vor und zurück wiegen\\
5--16 	& 3x Wiederholen
\danceeasymarker
\danceinstructionsend

\newpage

%%%%%%%%%%%%%%%%%%%%%%%%%%%%%%%%%%%%%%%%%%%%%%%%%%%%%%%%%%%%%%%%%%%%%%%%%%%%%%%%
% Traubentritt
%%%%%%%%%%%%%%%%%%%%%%%%%%%%%%%%%%%%%%%%%%%%%%%%%%%%%%%%%%%%%%%%%%%%%%%%%%%%%%%%

\dancename{Traubentritt}
\iffalse\subsection{Traubentritt}\fi % fool texstudio into displaying subsections
% \origininfo{}{}{}
\danceinfo{Longway for as many as will, keine Paarung}
\danceinstructionsbegin
1--8 	& Vier Simpel nach oben, links und rechts, auf 8 umdrehen\\
9--16 	& Vier Simpel zurück nach unten, zueinander wenden\\
\danceinstructionsel
1--4	& Referenz der Herren zu den Damen\\
5--8 	& Referenz der Damen zu den Herren\\
9--12 	& Referenz alle nach schräg Links\\
13--16 	& Referenz alle nach schräg Rechts\\
\danceinstructionsel
1--6 	& Die Dame dreht dreimal unter der rechten Hand des Herren\\
7--8	& Referenz beider\\
9--12	& Der hinterste Herr läuft durch die Gasse an die Spitze, alle anderen Herren rücken mit zwei Anstellschritten auf
\danceeasymarker
\danceinstructionsend
\newpage

%%%%%%%%%%%%%%%%%%%%%%%%%%%%%%%%%%%%%%%%%%%%%%%%%%%%%%%%%%%%%%%%%%%%%%%%%%%%%%%%
% Empty Page in the End for Layouting Multipage A4 Pages
%%%%%%%%%%%%%%%%%%%%%%%%%%%%%%%%%%%%%%%%%%%%%%%%%%%%%%%%%%%%%%%%%%%%%%%%%%%%%%%%

\newpage
% remove the background
\ClearShipoutPictureBG{}
\null  % to have some content (otherwise the page would not be displayed)
  
\end{document}
