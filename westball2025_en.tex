\documentclass[
	12pt,
	%draft, % show overfull hboxes
	]{scrartcl}

\usepackage[    % set up dimensions
	a6paper, 
	landscape,
	left=1.45cm,   
	right=1.45cm,  
	top=1.6cm,
	bottom=1.6cm, 
	%showframe,
	nomarginpar,  % this removes the right margin, centering the text in the middle
	noheadfoot,  % this removes the head and foot margin
	]{geometry}

%\pagestyle{empty}

\newif\iffast
\fastfalse

\usepackage{graphicx}
\usepackage{tikz}
\usetikzlibrary{external}
\usepackage{eso-pic}

% Fonts and encodings
\usepackage[nf]{coelacanth}
\usepackage[T1]{fontenc}

\usepackage{longtable} % Um Tabellen zu machen, die mehrere Seiten lang sind und einen passenden Seitenumbruch haben.
\usepackage{multicol}
%\usepackage[singlespacing]{setspace} % Zeilenabstand
\usepackage{nicefrac} % für besser aussehende Brüche

\usepackage[object=vectorian]{pgfornament}

% pstricks vs pgf/tikz:
% Originally we used pstricks to draw the background ornaments.
% This had the disadvantage of having to use `latex -> dvips -> ps2pdf` with the correct options to get the background
% graphics to display in the right orientation. The necessary options also varied between MiKTeX and TeXLive :-(.
%
% Since i have more experience with tikz, i decided to use that, because it was easier to position the
% elements (they weren't placed quite correctly with the pstricks version.).

% pgfornament vs eps files
% I wanted to use pgfornament at first because it has everything necessary in one easy package.
% Unfortunately the graphics in pgfornament end up having quite thick strokes. This does not look good at the scales
% we're using the patterns.
% So i downloaded the vectorian files and extracted the patterns manually with Inkscape. This probably leads to a slowdown though.


\AddToShipoutPictureBG{%
\begin{tikzpicture}[remember picture, overlay]
	% pgfornament does give very simple and nice vectorian graphics for tikz.
	% The problem is that the lines are too thick and figure 41 is also slightly cropped.
	% pst-vectorian works very well.
	% we're now using the original files from vectorian exported as eps file
	
	% DEBUGGING grid
	%\draw[line width=0.75pt, color=teal, step=.5cm, xshift=0.08cm, yshift=0.43cm] (current page.south west) grid ++(14.85cm,10.5cm);
	% DEBUGGING corners
	%\node [circle, fill=red, minimum size=1pt, xshift=0.0cm, yshift=-0.0cm, scale=0.2] at (current page.north west) {};
	%\node [circle, fill=red, minimum size=1pt, xshift=0.5cm, yshift=-0.5cm, scale=0.2] at (current page.north west) {};
	
	% TODO: This slows down compilation a lot.
	%       Maybe find a faster way to do this.
	\iffast
	\else
	\node[anchor=north west, xshift=0.5cm, yshift=-0.5cm, inner sep=0] at (current page.north west){\includegraphics[scale=0.8, angle=0, origin=c]{img/corner_black}};
	\node[anchor=north east, xshift=-0.5cm, yshift=-0.5cm, inner sep=0] at (current page.north east){\includegraphics[scale=0.8, angle=0, origin=c]{img/corner2_black}};
	\node[anchor=south west, xshift=0.5cm, yshift=0.5cm, inner sep=0] at (current page.south west){\includegraphics[scale=0.8, angle=180, origin=c]{img/corner2_black}};
	\node[anchor=south east, xshift=-0.5cm, yshift=0.5cm, inner sep=0] at (current page.south east){\includegraphics[scale=0.8, angle=180, origin=c]{img/corner_black}};
	\fi
	
	
	% Original pgfornament style:
%	\node[anchor=north west, xshift=0.5cm, yshift=-0.5cm,inner sep=0] at (current page.north west){\pgfornament[scale=0.5]{41}};
%	\node[anchor=north east, xshift=-0.5cm, yshift=-0.5cm, inner sep=0] at (current page.north east){\pgfornament[scale=0.5,symmetry=v]{41}};
%	\node[anchor=south west, xshift=0.5cm, yshift=0.5cm, inner sep=0] at (current page.south west){\pgfornament[scale=0.5,symmetry=h]{41}};
%	\node[anchor=south east, xshift=-0.5cm, yshift=0.5cm, inner sep=0] at (current page.south east){\pgfornament[scale=0.5,symmetry=c]{41}};
	
	% Background as an image, for example a parchment like texture
	%\tikz[remember picture, overlay] \node[opacity=0.5, inner sep=0pt] at (current page.center){\includegraphics[width=\paperwidth,height=\paperheight]{example-image}};
	
\end{tikzpicture}
}

\newcommand{\danceinfo}[1]{%
	{\raggedleft\footnotesize{#1}\par}
}
\newcommand{\dancename}[1]{
	% Dimension are manually chosen so they don't collide with the ornaments in the corners.	
	\begin{tikzpicture}[remember picture, overlay]
%		\node[] at (0, 0) (title) {Text};		
		\node[anchor=north, yshift=-0.5cm, inner sep=0, text width=10.5cm, align=center] at (current page.north){\Huge{#1}};
	\end{tikzpicture}
}
\newcommand{\origininfo}[3]{
	% Dimension are manually chosen so they don't collide with the ornaments in the corners.
	\begin{tikzpicture}[remember picture, overlay]
		% setting the font size in the node options, since it doesn't seem to 
		% take effect when used in the node text. 
		\node[anchor=south west, xshift=1.3cm, yshift=1.15cm, inner sep=0, align=left, font={\scriptsize}] at (current page.south west){Choreographie:\\{#1}};
		\node[anchor=south east, xshift=-1.2cm, yshift=1.15cm, inner sep=0, align=right, font={\scriptsize}] at (current page.south east){Musik:\\ {#2}\\ {#3}};
	\end{tikzpicture}
}
\newcommand{\danceeasymarker}{
	\iffast
	\else
		\begin{tikzpicture}[remember picture, overlay]
			\node[anchor=south, xshift=0cm, yshift=0.5cm, inner sep=0] at (current page.south){\pgfornament[width=4cm]{80}};
		\end{tikzpicture}
	\fi
}
\newcommand{\dancemediummarker}{
	\iffast
	\else
		\begin{tikzpicture}[remember picture, overlay]
			\node[anchor=south, xshift=0cm, yshift=0.5cm, inner sep=0] at (current page.south){\pgfornament[width=5cm]{83}};
		\end{tikzpicture}
	\fi
}
\newcommand{\dancedifficultmarker}{
	\iffast
	\else
		\begin{tikzpicture}[remember picture, overlay]
			\node[anchor=south, xshift=0cm, yshift=0.4cm, inner sep=0] at (current page.south){\pgfornament[width=5cm]{84}};
		\end{tikzpicture}
	\fi
}
\newcommand{\danceinstructionsbegin}{\begin{longtable}{p{1cm}p{9.8cm}}}
\newcommand{\danceinstructionsend}{\end{longtable}}
\newcommand{\danceinstructionsel}{~ & ~ \\}

\setlength{\topskip}{12pt}
\setlength{\parskip}{0pt}

\pagestyle{empty}

\begin{document}

%%%%%%%%%%%%%%%%%%%%%%%%%%%%%%%%%%%%%%%%%%%%%%%%%%%%%%%%%%%%%%%%%%%%%%%%%%%%%%%%
% A' Rovin
%%%%%%%%%%%%%%%%%%%%%%%%%%%%%%%%%%%%%%%%%%%%%%%%%%%%%%%%%%%%%%%%%%%%%%%%%%%%%%%%

\iffalse\subsection{A' Rovin}\fi % fool texstudio into displaying subsections
\dancename{A' Rovin}
\origininfo{Cecil Sharp }{A-Roving / Faithless Nancy Dawson}{Traditional/Anna Bidder}
\danceinfo{Longway for as many as will}
\danceinstructionsbegin
1 -- 8 	& P1 Lead Down \& Cast Up (vividly jumping)\\
9 -- 16	& P2 Lead Up \& Cast Down (vividly jumping)\\
\danceinstructionsel
1 -- 8 	& Dos-á-dos\\
\danceinstructionsel
1 -- 2	& Right-shouldered hop apart \& \nicefrac{1}{2} turn\\
3 -- 4  & Left-shouldered hop apart\\
5 -- 8  & Turn in right-shouldered towards each other\\
\danceinstructionsel
1 -- 8 	& Reach out right hand for \nicefrac{3}{4} \textbf{Chain}
\danceinstructionsend
\danceeasymarker

\newpage

%%%%%%%%%%%%%%%%%%%%%%%%%%%%%%%%%%%%%%%%%%%%%%%%%%%%%%%%%%%%%%%%%%%%%%%%%%%%%%%%
% Blaue Flagge
%%%%%%%%%%%%%%%%%%%%%%%%%%%%%%%%%%%%%%%%%%%%%%%%%%%%%%%%%%%%%%%%%%%%%%%%%%%%%%%%

\dancename{Blue Flag}
\iffalse\subsection{Blue Flag}\fi % fool texstudio into displaying subsections
\danceinfo{Circle, Herr in, Dame out, \nicefrac{3}{4} Beat, "Up" on 1 \\Face partner, hold right hands}
%\origininfo{???}{???}{}
\danceinstructionsbegin
1--6 	& Balance together, apart\\
7--12 	& Change places while turning Dame\\
1--12 	& Repeat\\
\danceinstructionsel
1--6 	& Balance together, apart\\
7--9 	& Herr step forth, turn Dame in to Kiekbuschfassung\\
10--12	& Together balance back with right foot\\
1--3 	& Together balance forth\\
4--6 	& Unwind Dame from Kiekbuschfassung, H switch hand\\
\danceinstructionsel
7--9 	& Both kick with inner foot against direction of circle\\
10--12 	& Dame turns below left arm of Herr to the next Herr
\danceeasymarker
\danceinstructionsend

\newpage

%%%%%%%%%%%%%%%%%%%%%%%%%%%%%%%%%%%%%%%%%%%%%%%%%%%%%%%%%%%%%%%%%%%%%%%%%%%%%%%%
% Branle du Rats
%%%%%%%%%%%%%%%%%%%%%%%%%%%%%%%%%%%%%%%%%%%%%%%%%%%%%%%%%%%%%%%%%%%%%%%%%%%%%%%%

\dancename{Branle du Rats}
\iffalse\subsection{Branle du Rats}\fi % fool texstudio into displaying subsections
%\origininfo{???}{???}{}
\danceinfo{Longway for as many as will (No Progress)}
\danceinstructionsbegin
        & \textbf{A} \\
1--8 	& All Double left, kick, Double right, kick\\
    	& \textit{Repeat A x3}\\
\danceinstructionsel
		& \textbf{B}\\
1--2 	& Jump left towards partner \\
3--4 	& Jump right towards partner to pass, le. shoulders come close\\
5--6 	& Jump left to change place with partner, back to back\\
7--8 	& Jump right to turn around \& stand on opposite ground line\\
        & \textit{Repeat B x3}\\
\danceeasymarker
\danceinstructionsend
\textit{Repeat all and adjust to the tempo of the music}

\newpage

%%%%%%%%%%%%%%%%%%%%%%%%%%%%%%%%%%%%%%%%%%%%%%%%%%%%%%%%%%%%%%%%%%%%%%%%%%%%%%%%
% Candles in the Dark
%%%%%%%%%%%%%%%%%%%%%%%%%%%%%%%%%%%%%%%%%%%%%%%%%%%%%%%%%%%%%%%%%%%%%%%%%%%%%%%%

\dancename{Candles in the Dark}
\origininfo{Loretta Holz (2006)}{Candles in the Dark}{von Jonathan Jensen}
\iffalse\subsection{Candles in the Dark}\fi % fool texstudio into displaying subsections
\danceinfo{Longway for as many as will\\¾ Takt, Waltz}
\danceinstructionsbegin
1--12 	& \textbf{H1} leads \textbf{D1} to Half-Figure Eight through the other pair\\
1--12 	& \textbf{H1} leads \textbf{D2} to Half-Figure Eight through the other pair\\
1--12 	& \textbf{H2} leads \textbf{D1} to Half-Figure Eight through the other pair\\
1--12 	& \textbf{H2} leads \textbf{D2} to Half-Figure Eight through the other pair\\
\danceinstructionsel
1--12 	& Mirror Dos-à-Dos on the line - Pair \textbf{2} inside \\
1--12 	& Whole set circle\\
\danceinstructionsel
1--12 	& Mirror Dos-à-Dos on the line – Pair \textbf{1} inside\\
1--12 	& 1 ½ Ronde with partner\\
\dancemediummarker
\danceinstructionsend

\newpage

%%%%%%%%%%%%%%%%%%%%%%%%%%%%%%%%%%%%%%%%%%%%%%%%%%%%%%%%%%%%%%%%%%%%%%%%%%%%%%%%
% Chapelloise
%%%%%%%%%%%%%%%%%%%%%%%%%%%%%%%%%%%%%%%%%%%%%%%%%%%%%%%%%%%%%%%%%%%%%%%%%%%%%%%%

\dancename{Chapelloise}
\iffalse\subsection{Chapelloise}\fi % fool texstudio into displaying subsections
\origininfo{Traditionell}{Jochen Eßrich}{Saitenhieb + Lebenswecker}
\danceinfo{Circle, Herren inside, counter-clockwise}
\danceinstructionsbegin
\textbf{I}&\\
1--4 	& 4 steps forth, turn around on 4 and change hands\\
5--8 	& 4 steps backwards in the same direction\\
1--8 	& \textit{Repeat} in the other direction\\
\textbf{II}&\\
1--2 	& Hop together, "flirt" look towards e.o.\\
3--4 	& Hop apart, look away\\
5--8 	& Switch places, Dame passes in front of the Herr\\
1--2 	& Hop together\\
3--4 	& Hop apart\\
5--8 	& Dame turns her left shoulder under his left arm\\
		& to move to the next Herr behind.
\danceeasymarker
\danceinstructionsend

\newpage

%%%%%%%%%%%%%%%%%%%%%%%%%%%%%%%%%%%%%%%%%%%%%%%%%%%%%%%%%%%%%%%%%%%%%%%%%%%%%%%%
% Circassian Circle
%%%%%%%%%%%%%%%%%%%%%%%%%%%%%%%%%%%%%%%%%%%%%%%%%%%%%%%%%%%%%%%%%%%%%%%%%%%%%%%%

\dancename{Circassian Circle}
\iffalse\subsection{Circassian Circle}\fi % fool texstudio into displaying subsections
\origininfo{Maud Karpeles (1918)}{Tanzfieber}{Teufelskreis}
\danceinfo{Circle, Herr-Dame, all hold hands}
\danceinstructionsbegin
1 -- 16 & All 2x Double forth and back\\
\danceinstructionsel
1 -- 8 	& Damen Double forth \& nack, (Herr clap)\\
1 -- 8  & Herren equally (Dame clap), face Dame to his left\\
\danceinstructionsel
1 -- 8  & Swingturn: Right feet touch. Right hands on partners right shoulder. Left hands held below. Kick with left foot to turn CW. Lean back for more fun.\\
9 -- 16 & Repeat counter-CW, end in Kiekbuschfassung \\
\danceinstructionsel
1 -- 12 & In Kiekbuschfassung walk on the circle line CW\\
13 -- 16& Unwind Dame under right arm of Herr to the right.\\
& \textit{Progress of Dame. All hold hands.}
\danceeasymarker
\danceinstructionsend


\newpage

%%%%%%%%%%%%%%%%%%%%%%%%%%%%%%%%%%%%%%%%%%%%%%%%%%%%%%%%%%%%%%%%%%%%%%%%%%%%%%%%
% Emperor of the Moon
%%%%%%%%%%%%%%%%%%%%%%%%%%%%%%%%%%%%%%%%%%%%%%%%%%%%%%%%%%%%%%%%%%%%%%%%%%%%%%%%

\dancename{Emperor of the Moon}
\iffalse\subsection{Emperor of the Moon}\fi % fool texstudio into displaying subsections
\origininfo{Playford}{Playford}{Emperor of the Moon}
\danceinfo{Longway for as many as will}%\\wird schneller}
\danceinstructionsbegin
1 -- 8 		& Set \& Turn left, go outside wide on the turn\\
9 -- 12 	& Double forth, meet again in the middle\\
13 -- 16 	& Set right\\
\danceinstructionsel
& \textit{The following fluently:}\\
1 -- 8 	& Pair 1 cast down, move towards place of Pair 2 \\
ab 3  	& P2 \textbf{first} crossing: to the counterpartners place\\
& \textit{Damen cross before Herren} \\
on 8    & Counterpartner hold both hands: \textbf{H1+D2} and \textbf{D1+H2} \\
1 -- 4 	& Counterpartners dance \nicefrac{1}{2} Ronde, with H1\&D1 inward \\
5 -- 8  & P2 \textbf{second} crossing, H1\&D1 lead D2\&H2 up \\
& with momentum from the Ronde (\textit{progress})\\
\dancemediummarker
\danceinstructionsend

\newpage

%%%%%%%%%%%%%%%%%%%%%%%%%%%%%%%%%%%%%%%%%%%%%%%%%%%%%%%%%%%%%%%%%%%%%%%%%%%%%%%%
% Heptathlon Jig
%%%%%%%%%%%%%%%%%%%%%%%%%%%%%%%%%%%%%%%%%%%%%%%%%%%%%%%%%%%%%%%%%%%%%%%%%%%%%%%%

\dancename{Heptathlon Jig}
\iffalse \subsection{Heptathlon Jig}\fi % fool texstudio into displaying subsections
\origininfo{Alan W. White}{I lost my Love / The Old Favorite}{}
\danceinfo{7 dancers, in an 'H' constellation}
\danceinstructionsbegin
1 -- 8 	& Star \textbf{RH} - 3 people in the \textbf{top-right} \\
1 -- 8 	& Star \textbf{LH} - 3 people in the \textbf{top-left}\\
1 -- 8 	& Star \textbf{RH} - 3 people in the \textbf{bottom-right}\\
1 -- 8 	& Star \textbf{LH} - 3 people in the \textbf{bottom-left}\\
\danceinstructionsel
1 -- 16 & Hey-Eight in the middle row\\
1 -- 8 	& Middle dances 1\nicefrac{1}{2} Ronde with \textbf{top right}\\
~		& \textit{or something else energetic that switches places}\\
1 -- 8 	& Set \& Turn left and\\
		& all outside positions move CW to the next spot\\
\danceeasymarker
\danceinstructionsend

\newpage

%%%%%%%%%%%%%%%%%%%%%%%%%%%%%%%%%%%%%%%%%%%%%%%%%%%%%%%%%%%%%%%%%%%%%%%%%%%%%%%%
% Hole in the Wall
%%%%%%%%%%%%%%%%%%%%%%%%%%%%%%%%%%%%%%%%%%%%%%%%%%%%%%%%%%%%%%%%%%%%%%%%%%%%%%%%

\dancename{Hole in the Wall}
\iffalse\subsection{Hole in the Wall}\fi % fool texstudio into displaying subsections
\origininfo{Playford}{Henry Purcell}{Hornpipe from Abdelazar}
\danceinfo{Longway for as many as will\\¾ Takt, Waltz steps}
\danceinstructionsbegin
1--12 	& Paar 1 Cast down \& Lead up\\
1--12 	& Paar 2 Cast up \& Lead down\\
~ & ~ \\
1--6 	& FC switch places, shortly stay in the middle\\
7--12	& SC switch places, kurz in der Mitte verharren\\
~ & ~ \\
1--6 	& \nicefrac{1}{2} Circle in the Set CW\\
7--12	& Pair 1 Cast down and pull Pair 2 up\\
\danceeasymarker
\danceinstructionsend

\newpage

%%%%%%%%%%%%%%%%%%%%%%%%%%%%%%%%%%%%%%%%%%%%%%%%%%%%%%%%%%%%%%%%%%%%%%%%%%%%%%%%
%	Indian Queen
%%%%%%%%%%%%%%%%%%%%%%%%%%%%%%%%%%%%%%%%%%%%%%%%%%%%%%%%%%%%%%%%%%%%%%%%%%%%%%%%

\dancename{Indian Queen}
\iffalse\subsection{Indian Queen}\fi % fool texstudio into displaying subsections
\origininfo{Playford}{Indian Queen}{}
\danceinfo{Longway for as many as will}
\danceinstructionsbegin
1 -- 4  & First Corner Set to partner, then to Contra\\
5 -- 8  & First Corner Turn right\\
9 -- 16 & First Corner Ronde\\
1 -- 8  & Second Corner Set \& Turn to partner, then to Contra\\
9 -- 16 & Second Corner Ronde\\
\danceinstructionsel
1 -- 8  & Mill \textbf{right}-handed, clap on the 8! \\
9 -- 16 & Mill \textbf{left}-handed, auf 8 Klatschen\\
\danceinstructionsel
1 -- 8  & Dos-à-dos\\
1 -- 8  & $\nicefrac{3}{4}$ Hand-Chain\\
\danceeasymarker
\danceinstructionsend

\newpage


%%%%%%%%%%%%%%%%%%%%%%%%%%%%%%%%%%%%%%%%%%%%%%%%%%%%%%%%%%%%%%%%%%%%%%%%%%%%%%%%
% Pavane d'Honneur
%%%%%%%%%%%%%%%%%%%%%%%%%%%%%%%%%%%%%%%%%%%%%%%%%%%%%%%%%%%%%%%%%%%%%%%%%%%%%%%%

\dancename{Pavane d'Honneur}
\iffalse\subsection{Pavane d'Honneur}\fi % fool texstudio into displaying subsections
%\origininfo{arg1}{arg2}{arg3}
\danceinfo{Row, Do not repeat Foreplay nor Finish, Face to the top}
\danceinstructionsbegin
1 -- 4 & (\textit{Foreplay:} \textbf{P1} to the \textbf{left} 2 Simple, \textbf{P2} to the \textbf{right} 2 Simple)\\
\danceinstructionsel
1 -- 8 	& Simple diagonally \textbf{forth} left, right, left, right\\
9 -- 16 & Simple diagonally \textbf{back} left, right, left, right\\
\danceinstructionsel
1 -- 8 	& \textit{Arrange:} \textbf{P1} to the \textbf{right} 4 Simple, \textbf{P2} to the \textbf{left} 4 Simple\\
9 -- 16 & Repeat back\\
\danceinstructionsel
1 -- 16 & Herr kneels down, Dame steps around him in eights\\
1 -- 16 & Herr rises, steps around Dame in eights, she turns too\\
\danceinstructionsel
1 -- 4 & (\textit{Finale:} \textbf{P1} to the \textbf{right} 2 Simple, \textbf{P2} to the \textbf{left} 2 Simple)
\dancemediummarker
\danceinstructionsend

\newpage

%%%%%%%%%%%%%%%%%%%%%%%%%%%%%%%%%%%%%%%%%%%%%%%%%%%%%%%%%%%%%%%%%%%%%%%%%%%%%%%%
% Red House
%%%%%%%%%%%%%%%%%%%%%%%%%%%%%%%%%%%%%%%%%%%%%%%%%%%%%%%%%%%%%%%%%%%%%%%%%%%%%%%%

\dancename{Red House}
\origininfo{Playford (1695)}{The Bare Necessities}{Red House}
\iffalse\subsection{Red House}\fi % fool texstudio into displaying subsections
\danceinfo{Longway for as many as will}

\danceinstructionsbegin
1 -- 8 	& \textbf{P1 Meet} \& Fallback, \textbf{P2 Fallback} \& Meet\\
1 -- 8  & P1 Set up, down, P1 Cast down, P2 lead up\\
1 -- 8 	& \textbf{P1 Meet} \& Fallback, \textbf{P2 Fallback} \& Meet\\
1 -- 8 	& P1 Set down, up, P1 Cast up, P2 lead down\\
\danceinstructionsel
& \textbf{Chasing:} \\
1 -- 12  & \textbf{H1 Cast out, D1 chases him}, he moves around Pair 2\\
13 -- 16 & When H1 stands behind D2: Pair 2 leads up \\
& and Pair 1 takes the places of Pair 2\\
1 -- 12  & \textbf{D2 Cast out, H2 chases her}, she moves around Pair 1 \\
13 -- 16 & When D2 stands behind H1: Pair 1 leads up \\
& and Pair 2 takes the places of Pair 1 \\
\danceinstructionsel
& \textbf{Hey:} \\
1 -- 16 	& Herr 2 starts a Hey-Eight with Pair 1\\
& first with Dame 1 right-shouldered\\
1 -- \textbf{12} & Dame 2 continues Hey with Paar 1 \textit{quickly} \\
& first with Herr 1 left-shouldered\\
13 -- 16 	& Paar 1 Cast down, Paar 2 lead up
\dancedifficultmarker
\danceinstructionsend


\newpage
%%%%%%%%%%%%%%%%%%%%%%%%%%%%%%%%%%%%%%%%%%%%%%%%%%%%%%%%%%%%%%%%%%%%%%%%%%%%%%%%
% Schiarazula Marazula
%%%%%%%%%%%%%%%%%%%%%%%%%%%%%%%%%%%%%%%%%%%%%%%%%%%%%%%%%%%%%%%%%%%%%%%%%%%%%%%%

\dancename{Schiarazula Marazula}
\iffalse\subsection{Schiarazula Marazula}\fi % fool texstudio into displaying subsections
\origininfo{Giorgio Mainerio (1574)}{Giorgio Mainerio}{Schiarazula Marazula}
\danceinfo{Circle, Herr/Dame hold hands}
\danceinstructionsbegin
1 -- 8	& Double left, Double right (crossing feet in front)\\
9 -- 16	& Repeat\\
\danceinstructionsel
1 -- 2	& Step \textbf{into the circle} (Herren right, Damen left)\\
		& Snap fingers on both hands towards \textbf{foreign} partner\\
3 -- 4	& Step with other foot \& snap towards \textbf{your} partner\\
5 -- 6	& Step with other foot, snap to \textbf{foreign} partner \\
7 -- 8	& Turn to outside \& clap 3 times (on 6, 6\textsuperscript{and}, and 7)\\
& \textit{and now back outside...}\\
1 -- 6 	& \textbf{Outside} to foreign, your, foreign Partner finger snap \\
7 -- 8	& Turn towards inside \& clap 3 times (on 6, 6\textsuperscript{and}, and 7)
\danceeasymarker
\danceinstructionsend


\newpage
%%%%%%%%%%%%%%%%%%%%%%%%%%%%%%%%%%%%%%%%%%%%%%%%%%%%%%%%%%%%%%%%%%%%%%%%%%%%%%%%
% 	Siege of St. Malo
%%%%%%%%%%%%%%%%%%%%%%%%%%%%%%%%%%%%%%%%%%%%%%%%%%%%%%%%%%%%%%%%%%%%%%%%%%%%%%%%

\dancename{Siege of St. Malo}
\iffalse\subsection{Siege of St. Malo}\fi % fool texstudio into displaying subsections
\origininfo{Christine Feyerabend}{Siege of St. Malo}{}
\danceinfo{Longway for as many as will}
\danceinstructionsbegin
\danceinstructionsel
1 -- 8  & First Corner Dos-à-dos\\
9 -- 16 & Second Corner Dos-à-dos\\
\danceinstructionsel
1 -- 8  & $\nicefrac{3}{4}$ Chain\\
\danceinstructionsel
1 -- 8  & Set \& Turn right
\danceeasymarker
\danceinstructionsend

\newpage


%%%%%%%%%%%%%%%%%%%%%%%%%%%%%%%%%%%%%%%%%%%%%%%%%%%%%%%%%%%%%%%%%%%%%%%%%%%%%%%%
% Tourdion
%%%%%%%%%%%%%%%%%%%%%%%%%%%%%%%%%%%%%%%%%%%%%%%%%%%%%%%%%%%%%%%%%%%%%%%%%%%%%%%%

\dancename{Tourdion}
\origininfo{Arbeau (1589)}{Tourdion}{Pierre Attaignant (1530)}
\iffalse\subsection{Tourdion}\fi % fool texstudio into displaying subsections
\danceinfo{Circe, Herr/Dame, hold hands}
\danceinstructionsbegin
1--4 	& Simple to the left, right, forth and back\\
5--16 	& 3x Repeat \\
\danceinstructionsel
1--2 	& Herrn lead the Dame on their \textbf{right to the place on their left},\\
3--4 	& then \textbf{simple} forth and back once\\
5--16 	& 3x Repeat \\
\danceinstructionsel
1--16 	& As in the beginning: Left, right, forth, back (x4)\\
%\danceinstructionsel
1--2 	& Damen lead the Herr on their \textbf{left to place on their right}, \\
3--4	& then \textbf{simple} forth and back once\\
5--16 	& 3x Repeat
\danceeasymarker
\danceinstructionsend

\newpage

%%%%%%%%%%%%%%%%%%%%%%%%%%%%%%%%%%%%%%%%%%%%%%%%%%%%%%%%%%%%%%%%%%%%%%%%%%%%%%%%
% Traubentritt
%%%%%%%%%%%%%%%%%%%%%%%%%%%%%%%%%%%%%%%%%%%%%%%%%%%%%%%%%%%%%%%%%%%%%%%%%%%%%%%%

\dancename{Traubentritt}
\iffalse\subsection{Traubentritt}\fi % fool texstudio into displaying subsections
% \origininfo{}{}{}
\danceinfo{Longway for as many as will, no pairs}
\danceinstructionsbegin
1--8 	& Four Simpel up, left and right, turn around on 8\\
9--16 	& Four Simpel back down, face each other\\
\danceinstructionsel
1--4	& Reference Herren to their Dame\\
5--8 	& Reference Damen to their Herr\\
9--12 	& Reference all diagonally to the left\\
13--16 	& Reference all diagonally to the right\\
\danceinstructionsel
1--6 	& The Dame turns three times under the Herr right hand \\
7--8	& Both reference each other\\
9--12	& Bottommost Herr runs through the Longway to the topmost place, when all Herren move down in 2 simple steps.
\danceeasymarker
\danceinstructionsend
\newpage

%%%%%%%%%%%%%%%%%%%%%%%%%%%%%%%%%%%%%%%%%%%%%%%%%%%%%%%%%%%%%%%%%%%%%%%%%%%%%%%%
% Empty Page in the End for Layouting Multipage A4 Pages
%%%%%%%%%%%%%%%%%%%%%%%%%%%%%%%%%%%%%%%%%%%%%%%%%%%%%%%%%%%%%%%%%%%%%%%%%%%%%%%%

\newpage
% remove the background
\ClearShipoutPictureBG{}
\null  % to have some content (otherwise the page would not be displayed)
  
\end{document}
